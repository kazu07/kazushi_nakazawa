\documentclass[uplatex,dvipdfmx,a4paper,11pt]{jsreport}
\usepackage{docmute}
\usepackage{graphicx}
%\usepackage[fleqn]{amsmath}
\usepackage[varg]{txfonts}
\begin{document}

\chapter*{研究業績}

\section*{原著論文}
\begin{enumerate}
    \item 中澤和司, 近藤和弘. 残響劣化した音声に対するノンレファレンス音声了解度推定における推
定精度向上. 電気学会論文誌C(電子・情報・システム部門誌), Vol. 143, No. 8, pp. 830–841,
2023.
\end{enumerate}


\section*{国際会議発表}
\begin{enumerate}

\item K. Nakazawa and K. Kondo, “ Non-Intrusive Speech Intelligibility Prediction of Speech With Additive Noise and Reverberation Using Multiple Deep Learning-Based Speech Enhancement,” in 2023 IEEE 12th Global Conference on Consumer Electronics (GCCE), Nara, Japan, 2023/10.

\item K. Nakazawa and K. Kondo, “Non-intrusive speech intelligibility prediction method for reverberant speech using neural network-based frequency segmentation and masking front-end,” in Inter-Noise 2023, Chiba, Japan, 2023/8.


\item K. Nakazawa and K. Kondo, “Non-intrusive speech intelligibility estimation using deep learning with speech enhancement and convolutional layers,” in 2022 Asia-Pacific Signal and Information Processing Association Annual Summit and Conference (APSIPA ASC), Chiang Mai, Thailand, 2022/11, pp. 1047–1052.

\item K. Nakazawa and K. Kondo, “Improving the accuracy of non-intrusive intelligibility estimation for reverberant speech using speech enhancement by optimizing the speech feature parameters,” in Proceedings of the 24rd International Congress on Acoustics, Gyeongju, South Korea, 2022/10, p. ABS–0721.

\item K. Nakazawa and K. Kondo, “Improvements to Non-Intrusive Intelligibility Prediction for Reverberant Speech,” in 2021 Asia-Pacific Signal and Information Processing Association Annual Summit and Conference (APSIPA ASC), Tokyo,Japan, 2021/12, pp. 608–613.

\item K. Nakazawa and K. Kondo, “The estimation of non-referential speech intelligibility using DNN De-reverberation with SRMR value,” in 2021 IEEE 10th Global Conference on Consumer Electronics (GCCE), Kyoto,Japan, 2021/10, pp. 518–519.

\item K. Nakazawa and K. Kondo, “On Non-Reference Speech Intelligibility Estimation Using DNN De-reverberation,” in 2020 IEEE 9th Global Conference on Consumer Electronics (GCCE), Kobe, Japan, 2020/10, pp. 721–722.

\item K. Nakazawa and K. Kondo, “De-reverberation using CNN for non-reference reverberant speech intelligibility estimation,” in Proceedings of the 23rd International Congress on Acoustics, Aachen, Germany, 2019/9, pp. 3098–3102.

\item K. Nakazawa and K. Kondo, “De-reverberation using DNN for Non-Reference Reverberant Speech Intelligibility Estimation,” in 2018 IEEE 7th Global Conference on Consumer Electronics (GCCE), Nara, Japan, 2018/10, pp. 349–350.

\end{enumerate}
\section*{国内会議発表}
\begin{enumerate}

\item 中澤和司, 近藤和弘,寺島裕貴,古川茂人,上村卓也, “複合音の非調和性に対する自然音分類に最適化された Deep neural networkの中間表現と心理実験の検出閾値の類似度分析”, 日本音響学会2023年秋季研究発表会, 日本,名古屋, 2023/9

\item 中澤和司,近藤和弘, “Audio Spectrogram Transformer を用いたNon-intrusive 音声了解度予測の検討”, 日本音響学会2023年春季研究発表会, 日本,オンライン, 2023/3, pp. 543–546.

\item K. Nakazawa, “A study of non-intrusive speech intelligibility estimation method for degraded speech using pseudo-references obtained by multiple speech enhancement DNNs,” in The 19th IEEE TOWERS in Kansai, Osaka, Japan, 2022/10, B6.

\item 中澤和司,近藤和弘, “音声強調による複数の疑似レファレンスを用いた劣化音声に対するノンレファレンス音声了解度推定方法の検討”, 日本音響2022年学会秋季研究発表会, 北海道,日本, 2022/9, pp. 469–472.

\item 中澤和司,近藤和弘, “CNN を用いたノンレファレンス音声了解度推定方法の検討”, 日本音響学会2022年春季研究発表会, オンライン, 2022/3, pp. 763–764.

\item K. Nakazawa, “A study of non-reference speech intelligibility estimation using CNN”, in The 19th IEEE TOWERS in Sendai, Sendai, Japan, Sep 11,2022, p.6.

\item 中澤和司,近藤和弘, “ノンレファレンス音声了解度推定に向けた周波数重み付け特徴量の検討”, 日本音響学会東北支部第4回東北地区音響学研究会, オンライン, Nov. 2021, pp. 4–5.

\item K. Nakazawa, “Speech intelligibility estimation using speech enhancement for reverberant speech”, in The 18th IEEE TOWERS, Online, 2021/11, A3-7.

\item 中澤和司,近藤和弘, “残響劣化音に対する DNN を用いたノンレファレンス音声了解度推定 -音声の個人性による影響を考慮して-”,  日本音響学会2021年春季研究発表会, オンライン, 2021/3, pp. 331–332.

\item 中澤和司,近藤和弘, “音声の個人性が残響劣化音に対する音声了解度に及ぼす影響の考察”,  日本音響学会東北支部第3回東北地区音響学研究会, オンライン, 2020/11, p. 15.

\item 中澤和司,近藤和弘, “DNN を用いた残響環境下における音声了解度の推定のための了解度測定方法の検討”, 日本音響学会2020年春季研究発表会, オンライン, 2020/2, pp. 537–538.

\item 中澤和司,近藤和弘, “残響劣化音に対する推定原音声を用いたノンレファレンス音声了解度推定”, 日本音響学会2020年秋季研究発表会, オンライン, 2020/9, pp. 311–314.

\item 中澤和司,近藤和弘, “残響環境下における音声了解度の評価”, 第2回 東北地区音響学研究会, 福島,日本, 2019/11, p. 13.

\item 中澤和司,近藤和弘, “CNNを用いた残響除去の検討”, 日本音響学会2019年春季研究発表会, 東京,日本, 2019/3, pp. 403–404.

\item 中澤和司,近藤和弘, “DNN を用いた残響除去における位相推定方法の検討”, 第1回 東北地区音響学研究会, 秋田,日本, 2018/11, p. 11.
\end{enumerate}
\end{document}
